\documentclass[11pt, oneside]{article}   	
\usepackage[margin=1in]{geometry}                		
\evensidemargin=0.25in  
\oddsidemargin=0.25in 
\textwidth=6.25in

\usepackage{fancyhdr}
\pagestyle{fancy}
\fancyhf{}
\fancyhead[R]{\thepage}


\usepackage{graphicx,color}	

%%% Here I define my colors.  You can go to  http://latexcolor.com/ to find a list of other colors you could add. 
\definecolor{green}{rgb}{0.0, 0.5, 0.0}			
\definecolor{red}{rgb}{1.0, 0.13, 0.32}
\definecolor{blue}{rgb}{0.0, 0.5, 1.0}	
\definecolor{orange}{rgb}{0.93, 0.57, 0.13}
\definecolor{purple}{rgb}{0.6, 0.33, 0.73}
\definecolor{brown}{rgb}{0.44, 0.26, 0.08}
					
\usepackage{amssymb, amsmath, amsfonts, amsthm, multicol}

\renewcommand\qedsymbol{$\blacksquare$}

\title{MPLNet Data Tool}
\author{Last updated}
\date{3 December 2023}



\begin{document}
\newcommand{\modular}[1]{\; \mbox{(mod $#1$)}}
\thispagestyle{empty}
\maketitle
\tableofcontents
\thispagestyle{empty}



\mbox{}
\newpage
\setcounter{page}{1}



\section{Description}
(PLACEHOLDER)
The objective of this project is to develop a functional graphical user interface 
for seamless communication and data retrieval from NASA's Micro Pulse Lidar Network 
(MPLNet). \\

The application is designed to streamline the downloading process, providing 
users with a user-friendly interface while eliminating the need for direct access 
to NASA's data site. \\

It enables comprehensive data selection from AppState's MPLNet 
site, facilitating the aggregation and transformation of data from netCDF4 format 
to CSV format. This enhancement aims to expedite further research endeavors.


%%%%%%%%%%%%%%%%%%%%%%%%%%%%%%%%%%%%%%%%%%%%%%%%%%%%%%%%%%%%%%%%%%
\newpage
%%%%%%%%%%%%%%%%%%%%%%%%%%%%%%%%%%%%%%%%%%%%%%%%%%%%%%%%%%%%%%%%%%

\section{buildDirList}

\textbf{Description} \\
        Returns a list of directories to be used for downloading the files\\

\noindent\textbf{Usage} \\
        buildDirList(years, months, days, data\_path='..\/data\/') \\

\noindent\textbf{Arguments} \\
        years (list): The list of years\\
        months (list): The list of months\\
        days (list): The list of days\\
        data\_path (str): The path to the data folder (default: '../data/')\\
            This is the path relative to the current working directory\\

\noindent\textbf{Returns} \\
        dirs (list): The list of directories to be downloaded\\

%%%%%%%%%%%%%%%%%%%%%%%%%%%%%%%%%%%%%%%%%%%%%%%%%%%%%%%%%%%%%%%%%%
%%%%%%%%%%%%%%%%%%%%%%%%%%%%%%%%%%%%%%%%%%%%%%%%%%%%%%%%%%%%%%%%%%

\section{buildFileList}

\textbf{Description} \\
        Returns a list of files to be downloaded based upon the user input\\

\noindent\textbf{Usage} \\
        buildFileList(years, months, days, fileTypes) \\

\noindent\textbf{Arguments} \\
        years (list): The list of years\\
        months (list): The list of months\\
        days (list): The list of days\\
        fileTypes (list): The list of file types\\

\noindent\textbf{Returns} \\
        files (list): The list of files to be downloaded\\

%%%%%%%%%%%%%%%%%%%%%%%%%%%%%%%%%%%%%%%%%%%%%%%%%%%%%%%%%%%%%%%%%%
\newpage
%%%%%%%%%%%%%%%%%%%%%%%%%%%%%%%%%%%%%%%%%%%%%%%%%%%%%%%%%%%%%%%%%%

\section{buildURLList}

\textbf{Description} \\
        Returns a list of urls to be downloaded based upon the user input\\

\noindent\textbf{Usage} \\
        buildURLList(years, months, days, fileTypes) \\

\noindent\textbf{Arguments} \\
        years (list): The list of years\\
        months (list): The list of months\\
        days (list): The list of days\\
        fileTypes (list): The list of file types\\

\noindent\textbf{Returns} \\
        urls (lis1): The list of urls to be downloaded\\

%%%%%%%%%%%%%%%%%%%%%%%%%%%%%%%%%%%%%%%%%%%%%%%%%%%%%%%%%%%%%%%%%%
%%%%%%%%%%%%%%%%%%%%%%%%%%%%%%%%%%%%%%%%%%%%%%%%%%%%%%%%%%%%%%%%%%

\section{buildSelectionList}

\textbf{Description} \\
        Returns a list of months, days, and files for selection\\

\noindent\textbf{Usage} \\
        buildSelectionList(years) \\

\noindent\textbf{Arguments} \\
        years (list): The years available for a given site\\

\noindent\textbf{Returns} \\
        months (list): The months available for a given site and year\\
        days (list): The days available for a given site, year, and month\\
        fileTypes (list): The files available for a given site, year, month, and day\\

%%%%%%%%%%%%%%%%%%%%%%%%%%%%%%%%%%%%%%%%%%%%%%%%%%%%%%%%%%%%%%%%%%
\newpage
%%%%%%%%%%%%%%%%%%%%%%%%%%%%%%%%%%%%%%%%%%%%%%%%%%%%%%%%%%%%%%%%%%

\section{create\_directory}

\textbf{Description} \\
        Returns the url for the mplnet file path\\

\noindent\textbf{Usage} \\
        create\_directory(path) \\

\noindent\textbf{Arguments} \\
        path (str): The path to be folder to be created\\

\noindent\textbf{Returns} \\
        None\\

%%%%%%%%%%%%%%%%%%%%%%%%%%%%%%%%%%%%%%%%%%%%%%%%%%%%%%%%%%%%%%%%%%
%%%%%%%%%%%%%%%%%%%%%%%%%%%%%%%%%%%%%%%%%%%%%%%%%%%%%%%%%%%%%%%%%%

\section{create\_export\_name}

\textbf{Description} \\

\noindent\textbf{Usage} \\
        create\_export\_name(selectedVars, variable)

\noindent\textbf{Arguments} \\
        selectedVars (object): object containing selected variables\\
        variable (str): name of variable to export\\

\noindent\textbf{Returns} \\
        filename (str): name of export file\\

%%%%%%%%%%%%%%%%%%%%%%%%%%%%%%%%%%%%%%%%%%%%%%%%%%%%%%%%%%%%%%%%%%
%%%%%%%%%%%%%%%%%%%%%%%%%%%%%%%%%%%%%%%%%%%%%%%%%%%%%%%%%%%%%%%%%%

\section{directory\_select}

\textbf{Description} \\
        Returns user selected directory from a list\\
        Depreciated as GUI will guide user selection\\

\noindent\textbf{Usage} \\
        directory\_select(name, options) \\

\noindent\textbf{Arguments} \\
        name (str): The name of the directory\\
        options (list): The options for the directory\\

\noindent\textbf{Returns} \\
        str: The user selected directory\\

%%%%%%%%%%%%%%%%%%%%%%%%%%%%%%%%%%%%%%%%%%%%%%%%%%%%%%%%%%%%%%%%%%
%%%%%%%%%%%%%%%%%%%%%%%%%%%%%%%%%%%%%%%%%%%%%%%%%%%%%%%%%%%%%%%%%%

\section{export}

\textbf{Description} \\
        Export data to csv file\\
        Uses current data directory to save csv file\\

\noindent\textbf{Usage} \\
        export(filename, files, variable) \\

\noindent\textbf{Arguments} \\
        filename (str): name of output csv file\\
        files (list): list of full file paths\\
        variable (str): name of variable to export\\

\noindent\textbf{Returns} \\
        None\\

%%%%%%%%%%%%%%%%%%%%%%%%%%%%%%%%%%%%%%%%%%%%%%%%%%%%%%%%%%%%%%%%%%
%%%%%%%%%%%%%%%%%%%%%%%%%%%%%%%%%%%%%%%%%%%%%%%%%%%%%%%%%%%%%%%%%%

\section{file\_select}

\textbf{Description} \\
        Returns user selected file from a list\\
        Depreciated as GUI will guide user selection\\

\noindent\textbf{Usage} \\
        file\_select(file\_list) \\

\noindent\textbf{Arguments} \\
        file\_list (list): The list of files\\

\noindent\textbf{Returns} \\
        str: The user selected file\\

%%%%%%%%%%%%%%%%%%%%%%%%%%%%%%%%%%%%%%%%%%%%%%%%%%%%%%%%%%%%%%%%%%
\newpage
%%%%%%%%%%%%%%%%%%%%%%%%%%%%%%%%%%%%%%%%%%%%%%%%%%%%%%%%%%%%%%%%%%

\section{get\_mplnet\_data}

\textbf{Description} \\
        Writes the mplnet data from file at url to file\_path\_name\\

\noindent\textbf{Usage} \\
        get\_mplnet\_data(url, file\_path\_name) \\

\noindent\textbf{Arguments} \\
        url (str): The url for the mplnet data file \\
        file\_path\_name (str): The full path and name of the file to be saved \\

\noindent\textbf{Returns} \\
        Boolean value indicating if the file was downloaded and saved successfully

%%%%%%%%%%%%%%%%%%%%%%%%%%%%%%%%%%%%%%%%%%%%%%%%%%%%%%%%%%%%%%%%%%
%%%%%%%%%%%%%%%%%%%%%%%%%%%%%%%%%%%%%%%%%%%%%%%%%%%%%%%%%%%%%%%%%%

\section{get\_mplnet\_html}

\textbf{Description} \\
        Returns the html for the mplnet address given \\
        Used to scrape for file variables to assist in downloading \\

\noindent\textbf{Usage} \\
        get\_mplnet\_html(url) \\

\noindent\textbf{Arguments} \\
        url (str): The url for the mplnet html \\

\noindent\textbf{Returns} \\
        requests.Response (object): The html for the mplnet address given \\

%%%%%%%%%%%%%%%%%%%%%%%%%%%%%%%%%%%%%%%%%%%%%%%%%%%%%%%%%%%%%%%%%%
%%%%%%%%%%%%%%%%%%%%%%%%%%%%%%%%%%%%%%%%%%%%%%%%%%%%%%%%%%%%%%%%%%

\section{get\_user\_path}

\textbf{Description} \\
        Returns the user specified path for the mplnet data\\

\noindent\textbf{Usage} \\
        get\_user\_path() \\

\noindent\textbf{Arguments} \\
        None\\

\noindent\textbf{Returns} \\
        str: The user specified local path for the mplnet data\\

%%%%%%%%%%%%%%%%%%%%%%%%%%%%%%%%%%%%%%%%%%%%%%%%%%%%%%%%%%%%%%%%%%
%%%%%%%%%%%%%%%%%%%%%%%%%%%%%%%%%%%%%%%%%%%%%%%%%%%%%%%%%%%%%%%%%%

\section{leap\_year}

\textbf{Description} \\
        Returns true if year is a leap year\\

\noindent\textbf{Usage} \\
        leap\_year(year) \\

\noindent\textbf{Arguments} \\
        year (int): The year to be checked\\

\noindent\textbf{Returns} \\
        bool: True if year is a leap year\\

%%%%%%%%%%%%%%%%%%%%%%%%%%%%%%%%%%%%%%%%%%%%%%%%%%%%%%%%%%%%%%%%%%
%%%%%%%%%%%%%%%%%%%%%%%%%%%%%%%%%%%%%%%%%%%%%%%%%%%%%%%%%%%%%%%%%%

\section{parse\_day}

\textbf{Description} \\
        Returns a list of the days available for a given site, year, and month\\

\noindent\textbf{Usage} \\
        parse\_day(html) \\

\noindent\textbf{Arguments} \\
        html (str): The html for the mplnet address given\\

\noindent\textbf{Returns} \\
        list: The days available for a given site, year, and month\\

%%%%%%%%%%%%%%%%%%%%%%%%%%%%%%%%%%%%%%%%%%%%%%%%%%%%%%%%%%%%%%%%%%
%%%%%%%%%%%%%%%%%%%%%%%%%%%%%%%%%%%%%%%%%%%%%%%%%%%%%%%%%%%%%%%%%%

\section{parse\_file}

\textbf{Description} \\
        Returns a list of the files available for a given site, year, month, and day\\

\noindent\textbf{Usage} \\
        parse\_file(html) \\

\noindent\textbf{Arguments} \\
        html (str): The html for the mplnet address given\\

\noindent\textbf{Returns} \\
        list: The files available for a given site, year, month, and day\\

%%%%%%%%%%%%%%%%%%%%%%%%%%%%%%%%%%%%%%%%%%%%%%%%%%%%%%%%%%%%%%%%%%
%%%%%%%%%%%%%%%%%%%%%%%%%%%%%%%%%%%%%%%%%%%%%%%%%%%%%%%%%%%%%%%%%%

\section{parse\_month}

\textbf{Description} \\
        Returns a list of the months available for a given site and year\\

\noindent\textbf{Usage} \\
        parse\_month(html) \\

\noindent\textbf{Arguments} \\
        html (str): The html for the mplnet address given\\

\noindent\textbf{Returns} \\
        list: The months available for a given site and year\\

%%%%%%%%%%%%%%%%%%%%%%%%%%%%%%%%%%%%%%%%%%%%%%%%%%%%%%%%%%%%%%%%%%
%%%%%%%%%%%%%%%%%%%%%%%%%%%%%%%%%%%%%%%%%%%%%%%%%%%%%%%%%%%%%%%%%%
\section{parse\_year}

\textbf{Description} \\
        Returns a list of the years available for a given site

\noindent\textbf{Usage} \\
        parse\_year(html) \\

\noindent\textbf{Arguments} \\
        html (str): The html for the mplnet address given \\

\noindent\textbf{Returns} \\
        list: The years available for a given site \\

%%%%%%%%%%%%%%%%%%%%%%%%%%%%%%%%%%%%%%%%%%%%%%%%%%%%%%%%%%%%%%%%%%
\newpage
%%%%%%%%%%%%%%%%%%%%%%%%%%%%%%%%%%%%%%%%%%%%%%%%%%%%%%%%%%%%%%%%%%

\section{FileVariables (class)}

\textbf{Description} \\
        Class to build and store file variables\\

\noindent\textbf{Usage} \\
        fileVar = FileVariables() \\

\noindent\textbf{Attributes} \\
        filetype (str): The file type\\
        filevars (list): The file variables\\
        nextIter (int): The next file variable to be filled\\
        selectedFileType (str): The selected file type\\
        selectedFileVars (list): The selected file variables\\

\noindent\textbf{Methods} \\
        setFileTypes(ft): Store the file types from the downloaded files\\
            into the filetype variable to allow user selection\\
        next(): Returns the next file variable to be filled\\
            based upon the user input\\
        peakNext(): Returns the next file variable to be filled\\
            based upon the user input\\
        storeCurrent(value): Returns the current file variable to be filled\\
            based upon the user input\\
        setFileVars(file): Returns the file variables for the selected file type\\
        printSelected(var): Prints the selected file variable\\

%%%%%%%%%%%%%%%%%%%%%%%%%%%%%%%%%%%%%%%%%%%%%%%%%%%%%%%%%%%%%%%%%%
%%%%%%%%%%%%%%%%%%%%%%%%%%%%%%%%%%%%%%%%%%%%%%%%%%%%%%%%%%%%%%%%%%

\section{FileVariables.next}

\textbf{Description} \\
        Returns the next file variable to be filled \\
        based upon the user input\\

\noindent\textbf{Usage} \\
        fileVar.next() \\

\noindent\textbf{Arguments} \\
        None\\

\noindent\textbf{Returns} \\
        The next file variable to be filled\\

%%%%%%%%%%%%%%%%%%%%%%%%%%%%%%%%%%%%%%%%%%%%%%%%%%%%%%%%%%%%%%%%%%
%%%%%%%%%%%%%%%%%%%%%%%%%%%%%%%%%%%%%%%%%%%%%%%%%%%%%%%%%%%%%%%%%%

\section{FileVariables.peakNext}

\textbf{Description} \\
        Returns the next file variable to be filled\\
        based upon the user input\\

\noindent\textbf{Usage} \\
        fileVar.peakNext() \\

\noindent\textbf{Arguments} \\
        None\\

\noindent\textbf{Returns} \\
        Boolean value indicating if there is a next file variable\\

%%%%%%%%%%%%%%%%%%%%%%%%%%%%%%%%%%%%%%%%%%%%%%%%%%%%%%%%%%%%%%%%%%
%%%%%%%%%%%%%%%%%%%%%%%%%%%%%%%%%%%%%%%%%%%%%%%%%%%%%%%%%%%%%%%%%%

\section{FileVariables.printSelected}

\textbf{Description} \\
        Prints the selected file variable\\

\noindent\textbf{Usage} \\
        fileVar.printSelected(var) \\

\noindent\textbf{Arguments} \\
        var (str): The selected file variable\\

\noindent\textbf{Returns} \\
        String with the selected file variable\\

%%%%%%%%%%%%%%%%%%%%%%%%%%%%%%%%%%%%%%%%%%%%%%%%%%%%%%%%%%%%%%%%%%
%%%%%%%%%%%%%%%%%%%%%%%%%%%%%%%%%%%%%%%%%%%%%%%%%%%%%%%%%%%%%%%%%%

\section{FileVariables.setFileTypes}

\textbf{Description} \\
        Store the file types from the downloaded files\\
        into the filetype variable to allow user selection\\

\noindent\textbf{Usage} \\
        fileVar.setFileTypes(ft) \\

\noindent\textbf{Arguments} \\
        ft (list): The list of file types\\

\noindent\textbf{Returns} \\
        None\\

%%%%%%%%%%%%%%%%%%%%%%%%%%%%%%%%%%%%%%%%%%%%%%%%%%%%%%%%%%%%%%%%%%
%%%%%%%%%%%%%%%%%%%%%%%%%%%%%%%%%%%%%%%%%%%%%%%%%%%%%%%%%%%%%%%%%%

\section{FileVariables.setFileVars}

\textbf{Description} \\
        Opens the netcdf file and extracts the variables and populates\\
        the filevars dictionary\\

\noindent\textbf{Usage} \\
        fileVar.setFileVars(file) \\

\noindent\textbf{Arguments} \\
        files (str): A full path to the netcdf file to be opened\\

\noindent\textbf{Returns} \\
        None\\

%%%%%%%%%%%%%%%%%%%%%%%%%%%%%%%%%%%%%%%%%%%%%%%%%%%%%%%%%%%%%%%%%%
%%%%%%%%%%%%%%%%%%%%%%%%%%%%%%%%%%%%%%%%%%%%%%%%%%%%%%%%%%%%%%%%%%

\section{FileVariables.storeCurrent}

\textbf{Description} \\
        Stores the current file variable to be filled \\
        based upon the user input and the nextIter variable\\

\noindent\textbf{Usage} \\
        fileVar.storeCurrent(value) \\

\noindent\textbf{Arguments} \\
        None\\

\noindent\textbf{Returns} \\
        None\\

%%%%%%%%%%%%%%%%%%%%%%%%%%%%%%%%%%%%%%%%%%%%%%%%%%%%%%%%%%%%%%%%%%
\newpage
%%%%%%%%%%%%%%%%%%%%%%%%%%%%%%%%%%%%%%%%%%%%%%%%%%%%%%%%%%%%%%%%%%

\section{SelectionVariables (class)}

\textbf{Description} \\
        Class to build and store selection variables\\

\noindent\textbf{Usage} \\
        selVar = SelectionVariables() \\

\noindent\textbf{Attributes} \\
        years (list): The list of years\\
        months (list): The list of months\\
        days (list): The list of days\\
        fileTypes (list): The list of file types\\
        selectedYears (list): The list of selected years\\
        selectedMonths (list): The list of selected months\\
        selectedDays (list): The list of selected days\\
        selectedFileTypes (list): The list of selected file types\\
        nextIter (int): The next selection variable to be filled\\

\noindent\textbf{Methods} \\
        getVars(html): Returns the years, months, days, and file types from the html\\
        setSelectVars(years, months, days, fileTypes): Sets the selection variables based upon the user input\\
        next(): Returns the next selection variable to be filled\\
            based upon the user input\\
        peakNext(): Returns the next selection variable to be filled\\
            based upon the user input\\
        storeCurrent(value): Returns the current selection variable to be filled\\
            based upon the user input\\
        prepDownload(): Returns the urls, directories, and files to be downloaded\\
        download(url, dir, file): Downloads the selected files\\
        checkSelection(): Checks to make sure a selection is made in each category\\
        reset(): Resets the nextIter variable to 0\\
        printSelected(): Prints the selected variables\\

%%%%%%%%%%%%%%%%%%%%%%%%%%%%%%%%%%%%%%%%%%%%%%%%%%%%%%%%%%%%%%%%%%
\newpage
%%%%%%%%%%%%%%%%%%%%%%%%%%%%%%%%%%%%%%%%%%%%%%%%%%%%%%%%%%%%%%%%%%

\section{SelectionVariables.checkSelection}

\textbf{Description} \\
        Checks to make sure a selection is made in each category\\

\noindent\textbf{Usage} \\
        selVar.checkSelection() \\

\noindent\textbf{Arguments} \\
        None\\

\noindent\textbf{Returns} \\
        Boolean value indicating if a selection is made in each category\\

%%%%%%%%%%%%%%%%%%%%%%%%%%%%%%%%%%%%%%%%%%%%%%%%%%%%%%%%%%%%%%%%%%
%%%%%%%%%%%%%%%%%%%%%%%%%%%%%%%%%%%%%%%%%%%%%%%%%%%%%%%%%%%%%%%%%%

\section{SelectionVariables.download}

\textbf{Description} \\
        Downloads the selected files\\

\noindent\textbf{Usage} \\
        selVar.download(url, dir, file) \\

\noindent\textbf{Arguments} \\
        url (str): The url for the mplnet data file\\
        dir (str): The full path of the file to be saved, not including the file name\\
        file (str): The name of the file to be saved\\

\noindent\textbf{Returns} \\
        String with the file name and status of the download\\

%%%%%%%%%%%%%%%%%%%%%%%%%%%%%%%%%%%%%%%%%%%%%%%%%%%%%%%%%%%%%%%%%%
%%%%%%%%%%%%%%%%%%%%%%%%%%%%%%%%%%%%%%%%%%%%%%%%%%%%%%%%%%%%%%%%%%

\section{SelectionVariables.getVars}

\textbf{Description} \\
        Builds the years, months, days, and file types from the html\\
        to be used in the selection of files to be downloaded\\

\noindent\textbf{Usage} \\
        selVar.getVars(html) \\

\noindent\textbf{Arguments} \\
        html (str): The html from the mplnet website\\

\noindent\textbf{Returns} \\
        None \\

%%%%%%%%%%%%%%%%%%%%%%%%%%%%%%%%%%%%%%%%%%%%%%%%%%%%%%%%%%%%%%%%%%
%%%%%%%%%%%%%%%%%%%%%%%%%%%%%%%%%%%%%%%%%%%%%%%%%%%%%%%%%%%%%%%%%%

\section{SelectionVariables.next}

\textbf{Description} \\
        Returns the next selection variable to be filled \\
        based upon the user input\\

\noindent\textbf{Usage} \\
        selVar.next() \\

\noindent\textbf{Arguments} \\
        None\\

\noindent\textbf{Returns} \\
        The next selection variable to be filled or None\\

%%%%%%%%%%%%%%%%%%%%%%%%%%%%%%%%%%%%%%%%%%%%%%%%%%%%%%%%%%%%%%%%%%
%%%%%%%%%%%%%%%%%%%%%%%%%%%%%%%%%%%%%%%%%%%%%%%%%%%%%%%%%%%%%%%%%%

\section{SelectionVariables.peakNext}

\textbf{Description} \\
        Returns the next selection variable to be filled\\
        based upon the user input\\

\noindent\textbf{Usage} \\
        selVar.peakNext() \\

\noindent\textbf{Arguments} \\
        None\\

\noindent\textbf{Returns} \\
        Boolean value indicating if there is a next selection variable\\

%%%%%%%%%%%%%%%%%%%%%%%%%%%%%%%%%%%%%%%%%%%%%%%%%%%%%%%%%%%%%%%%%%
%%%%%%%%%%%%%%%%%%%%%%%%%%%%%%%%%%%%%%%%%%%%%%%%%%%%%%%%%%%%%%%%%%

\section{SelectionVariables.prepDownload}

\textbf{Description} \\
        Prepares a tuple the selected files to be downloaded including
        the urls, directories, and files

\noindent\textbf{Usage} \\
        selVar.prepDownload() \\

\noindent\textbf{Arguments} \\
        None\\

\noindent\textbf{Returns} \\
        Tuple of lists containing the urls, dirs, and, files to be downloaded\\

%%%%%%%%%%%%%%%%%%%%%%%%%%%%%%%%%%%%%%%%%%%%%%%%%%%%%%%%%%%%%%%%%%
%%%%%%%%%%%%%%%%%%%%%%%%%%%%%%%%%%%%%%%%%%%%%%%%%%%%%%%%%%%%%%%%%%

\section{SelectionVariables.printSelected}

\textbf{Description} \\
        Prints the selected variables\\

\noindent\textbf{Usage} \\
        selVar.printSelected() \\

\noindent\textbf{Arguments} \\
        None\\

\noindent\textbf{Returns} \\
        String with the selected variables\\

%%%%%%%%%%%%%%%%%%%%%%%%%%%%%%%%%%%%%%%%%%%%%%%%%%%%%%%%%%%%%%%%%%
%%%%%%%%%%%%%%%%%%%%%%%%%%%%%%%%%%%%%%%%%%%%%%%%%%%%%%%%%%%%%%%%%%

\section{SelectionVariables.reset}

\textbf{Description} \\
        Resets the nextIter variable to 0 and clears the selected values\\

\noindent\textbf{Usage} \\
        selVar.reset() \\

\noindent\textbf{Arguments} \\
        None\\

\noindent\textbf{Returns} \\
        None\\

%%%%%%%%%%%%%%%%%%%%%%%%%%%%%%%%%%%%%%%%%%%%%%%%%%%%%%%%%%%%%%%%%%
%%%%%%%%%%%%%%%%%%%%%%%%%%%%%%%%%%%%%%%%%%%%%%%%%%%%%%%%%%%%%%%%%%

\section{SelectionVariables.storeCurrent}

\textbf{Description} \\
        Returns the current selection variable to be filled \\
        based upon the user input\\

\noindent\textbf{Usage} \\
        selVar.storeCurrent(value) \\

\noindent\textbf{Arguments} \\
        value (list): The current selection values to be stored\\

\noindent\textbf{Returns} \\
        None\\

%%%%%%%%%%%%%%%%%%%%%%%%%%%%%%%%%%%%%%%%%%%%%%%%%%%%%%%%%%%%%%%%%%
%%%%%%%%%%%%%%%%%%%%%%%%%%%%%%%%%%%%%%%%%%%%%%%%%%%%%%%%%%%%%%%%%% 
           
\end{document}  
